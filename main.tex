\documentclass{article}
\usepackage[margin=0.5in]{geometry}
\usepackage[utf8]{inputenc}

\usepackage{amsmath}
\usepackage{amsthm}
\usepackage{amssymb}
\usepackage{enumerate}
\usepackage{chngcntr}
\usepackage{mathtools}
\usepackage{enumitem}
\usepackage{listings}
\usepackage{hyperref}
\usepackage[dvipsnames]{xcolor}
\usepackage[bb=boondox]{mathalfa}
\newcommand{\Z}{\mathbb{Z}}
\newcommand{\C}{\mathbb{C}}
\newcommand{\HH}{\mathbb{H}}
\newcommand{\Q}{\mathbb{Q}}
\newcommand{\R}{\mathbb{R}}
\newcommand{\N}{\mathbb{N}}
\newcommand{\F}{\mathbb{F}}
\newcommand{\qbinom}{\genfrac{[}{]}{0pt}{}}
\DeclarePairedDelimiter\ceil{\lceil}{\rceil}
\DeclarePairedDelimiter\floor{\lfloor}{\rfloor}

\usepackage[dvipsnames]{xcolor}

\newtheorem{theorem}{Theorem}
\newtheorem{corollary}{Corollary}[theorem] 
\newtheorem{lemma}[theorem]{Lemma} 
\newtheorem{proposition}{Proposition}
\newcommand{\greenparagraph}[1]{\textcolor{ForestGreen}{\textbf{#1}}}

\theoremstyle{definition}
\newtheorem{definition}{Definition}[section]
\theoremstyle{remark}
\newtheorem*{remark}{Remark}
\theoremstyle{remark}
\newtheorem*{note}{Note}
\theoremstyle{definition}
\newtheorem{example}{Example}[definition]
\newcounter{exercise}[subsection]
\newenvironment{exercise}{\refstepcounter{exercise}\textbf{Exercise~\theexercise}}{}
\counterwithin*{equation}{section}
\counterwithin*{equation}{subsection}
\lstset{
  basicstyle=\ttfamily,
  mathescape
}
\title{Rust Course notes}
\author{Razafy Rindra, "Hagamena"}

\begin{document}

\maketitle
\tableofcontents
\newpage
\section{Introduction}
Some quick notes from the Rust book, and the "Let's Get Rusty" \href{https://youtube.com/playlist?list=PLai5B987bZ9CoVR-QEIN9foz4QCJ0H2Y8}{online course}
\section{Common Programming Concepts in Rust}
\subsection{Variables and Mutability}
\subsubsection{Mutable variables} Variables in Rust are immmutable by default. In order to createe a mutable variable, we need to add in 'mut' in front of the name like so:
\begin{verbatim}
fn main() {
    let mut x = 5;
    println!("The value of x is: {x}");
    x = 6;
    println!("The value of x is: {x}");
}
\end{verbatim}
\subsubsection{Constants} We can also declare constants in Rust like so:\begin{verbatim}
    const MY_NUMBER: u32 = 100_000;
\end{verbatim}

Constants unlike variables can't be mutable, need to be type annotated and they can't be assigned to return value of a function or any value computed at run time.

\subsubsection{Shadowing}


\subsection{Data Types}
\begin{definition}
    \textbf{Scalar data types} represent a single value, \textbf{Compound data types} represent a goup of values
\end{definition}
Types of scalar data types\begin{enumerate}
    \item Integers, they can be 8,16,32,64,128 bit signed or unsigned integers.
    \item Floating-point numbers
    \item Booleans
    \item Character, they represent unicode character
\end{enumerate}
Type of compound data types\begin{enumerate}
    \item Tuples; a fixed size array of data that can be of different values.
    \begin{verbatim}
        let tup = ("Let's Get Rusty!",100_000);
    \end{verbatim}
    \item Arrays, in Rust they are fixed length.
\end{enumerate}
\subsection{Functions}
Functions are defined with an fn keyword like so:\begin{verbatim}
    fn my_function(x: i32,y: i32){
         println!("The value of x is: {}", x);
         println!("The value of y is: {}", y);
    }
\end{verbatim}

A piece in code in Rust are either a statement or an expression. Statements perform some action but do not return any value, whilst expressions returns values. 
\begin{verbatim}
    fn my_function(x: i32,y: i32) -> i32{
         println!("The value of x is: {}", x);
         println!("The value of y is: {}", y);
         x+y
    }
\end{verbatim}

In this function, the println are statements since they don't return anything while 'x+y' is an expression, it is returned by the function (note the last expression of a function is implicitely returned). 

\subsection{Control Flow}
\subsubsection{if statements}
As in other programming languages:\begin{verbatim}
 fn main() {
    let number = 3;

    if number < 5 {
        println!("first condition was true");
    } else if number < 22{
        println!("second condition was true.")
    }else{
        println!("condition was false");
    }
}
   
\end{verbatim}
Note, the condition must be a boolean. We can also set a "if-else" statement in a let statemetn.\begin{verbatim}
    fn main() {
    let condition = true;
    let number = if condition { 5 } else { 6 };

    println!("The value of number is: {number}");
}
\end{verbatim}
\subsubsection{Loops}
We can create loops with the 'loop' keyword, which will execute the code in it until we call break.\begin{verbatim}
    let mut counter = 0;
    let result = loop{
       counter += 1;
       if counter == 10{
            break counter;
       }
    };
\end{verbatim}

We can also use the while statement:\begin{verbatim}
    let mut number = 3;
    while number != 0{
        println!("{}!", number);
        number -= 1;
    }
\end{verbatim}

Finally the third type of loop we can create are 'for loops'\begin{verbatim}
    let a = [10,20,30,40,50];
    
    for element in a.iter{
        println!("The value is: {}", element);
    }
\end{verbatim}

We can also loop over a range:\begin{verbatim}
    for number in 1..4{
        println!("{}!", number);
    }
\end{verbatim}
The last number of the range is excluded.
\end{document}